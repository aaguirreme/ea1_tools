%%%%%%%%%%%%%%%%%%%%%%%%%%%%%%%%%%%%%%%%%%%%%%%%%%%%%%%%%%%%%%%%%%%%%%%%%%%%%%%%
%
% Name:   flashcards_laplace_large
%
% Author: Andres M. Aguirre-Mesa
%         Ph.D student, Mechanical Engineering
%         University of Texas at San Antonio
%
% Date:   03/21/2019
%
%%%%%%%%%%%%%%%%%%%%%%%%%%%%%%%%%%%%%%%%%%%%%%%%%%%%%%%%%%%%%%%%%%%%%%%%%%%%%%%%

%\documentclass[grid, avery5371]{flashcards}
\documentclass[grid, custom]{flashcards}

\usepackage{amsmath}
\usepackage{tikz}

\begin{document}

  %\cardfrontstyle[\large]{headings}
  %\cardbackstyle[\large]{plain}
  \cardfrontstyle[\LARGE]{headings}
  \cardbackstyle[\LARGE]{plain}

  \cardfrontfoot{EGR2323 - Engineering Analysis 1}


  % Definition of the Laplace Transform

  %1
  \begin{flashcard}{ Integral definition of $ \mathcal{L} \{ f(t) \} $ }
    \[
      \mathcal{L} \{ f(t) \}  = \int_0^\infty e^{-st} f(t) \, dt
    \]
  \end{flashcard}


  % Transforms of basic functions

  %2
  \begin{flashcard}{ \[ \mathcal{L} \{ 1 \} \] }
    \[
      \mathcal{L} \{ 1 \}  = \frac{1}{s}
    \]
  \end{flashcard}

  %3
  \begin{flashcard}{ \[ \mathcal{L} \{ t \} \] }
    \[
      \mathcal{L} \{ t \}  = \frac{ 1 }{ s^2 }
    \]
  \end{flashcard}

  %4
  \begin{flashcard}{ \[ \mathcal{L} \{ t^n \}, \quad n=1,2,3,\ldots \] }
    \[
      \mathcal{L} \{ t^n \}  = \frac{ n! }{ s^{n + 1} }, \quad n=1,2,3,\ldots
    \]
  \end{flashcard}

  %5 
  \begin{flashcard}{ \[ \mathcal{L} \{ e^{at} \} \] }
    \[
      \mathcal{L} \{ e^{at} \} = \frac{ 1 }{ s - a }
    \]
  \end{flashcard}

  %6
  \begin{flashcard}{\[ \mathcal{L} \{ \sin kt \} \] }
    \[
      \mathcal{L} \{ \sin kt \}  = \frac{ k }{ s^2 + k^2 }
    \]
  \end{flashcard}

  %7
  \begin{flashcard}{ \[ \mathcal{L} \{ \cos kt \} \] }
    \[
      \mathcal{L} \{ \cos kt \}  = \frac{ s }{ s^2 + k^2 }
    \]
  \end{flashcard}

  %8
  \begin{flashcard}{ \[ \mathcal{L} \{ \sinh kt \} \] }
    \[
      \mathcal{L} \{ \sinh kt \}  = \frac{ k }{ s^2 - k^2 }
    \]
  \end{flashcard}

  %9
  \begin{flashcard}{ \[ \mathcal{L} \{ \cosh kt \} \] }
    \[
      \mathcal{L} \{ \cosh kt \}  = \frac{ s }{ s^2 - k^2 }
    \]
  \end{flashcard}


  % Inverse transforms

  %10
  \begin{flashcard}{ \[ \mathcal{L}^{-1} \left \{ \frac{1}{s} \right \} \] }
    \[
      \mathcal{L}^{-1} \left \{ \frac{1}{s} \right \}  = 1
    \]
  \end{flashcard}

  %11
  \begin{flashcard}{ \[ \mathcal{L}^{-1} \left \{ \frac{1}{s^2} \right \} \] }
    \[
      \mathcal{L}^{-1} \left \{ \frac{1}{s^2} \right \}  = t
    \]
  \end{flashcard}

  %12
  \begin{flashcard}{ \[ \mathcal{L}^{-1} 
    \left \{ \frac{ n! }{ s^{n + 1} } \right \}, \quad n=1,2,3,\ldots \] }
    \[
      \mathcal{L}^{-1} \left \{ \frac{ n! }{ s^{n + 1} } \right \}  = t^n,
      n=1,2,3,\ldots
    \]
  \end{flashcard}

  %13
  \begin{flashcard}{ \[ \mathcal{L}^{-1} 
    \left \{ \frac{ 1 }{ s - a } \right \} \] }
    \[
      \mathcal{L}^{-1} \left \{ \frac{ 1 }{ s - a } \right \}  = e^{at}
    \]
  \end{flashcard}

  %14
  \begin{flashcard}{ \[ \mathcal{L}^{-1} 
    \left \{ \frac{ k }{ s^2 + k^2 } \right \}  \] }
    \[
      \mathcal{L}^{-1} \left \{ \frac{ k }{ s^2 + k^2 } \right \}  = \sin kt
    \]
  \end{flashcard}

  %15
  \begin{flashcard}{\[ \mathcal{L}^{-1} 
    \left \{ \frac{ s }{ s^2 + k^2 } \right \} \] }
    \[
      \mathcal{L}^{-1} \left \{ \frac{ s }{ s^2 + k^2 } \right \}  = \cos kt
    \]
  \end{flashcard}

  %16
  \begin{flashcard}{ \[ \mathcal{L}^{-1} 
    \left \{ \frac{ k }{ s^2 - k^2 } \right \}  \] }
    \[
      \mathcal{L}^{-1} \left \{ \frac{ k }{ s^2 - k^2 } \right \}  = \sinh kt
    \]
  \end{flashcard}

  %17
  \begin{flashcard}{ \[ \mathcal{L}^{-1} 
    \left \{ \frac{ s }{ s^2 - k^2 } \right \} \] }
    \[
      \mathcal{L}^{-1} \left \{ \frac{ s }{ s^2 - k^2 } \right \}  = \cosh kt
    \]
  \end{flashcard}


  % Transforms of derivatives

  %18
  \begin{flashcard}{ \[ \mathcal{L} \{ f'(t) \} \] }
    \begin{align*}
      \mathcal{L}\{f'(t)\} & = s \, \mathcal{L}\{ f(t) \} - f(0) \\
                           & = s \, F(s)-f(0)
    \end{align*}
  \end{flashcard}

  %19
  \begin{flashcard}{ \[ \mathcal{L} \{ f''(t) \} \] }
    \begin{align*}
      \mathcal{L}\{f''(t)\} & = s^{2} \mathcal{L}\{f(t)\} -s\,f(0) - f'(0)\\
                            & = s^{2} F(s) -s\,f(0) - f'(0)
    \end{align*}
  \end{flashcard}


  % Translations

  %20
  \begin{flashcard}{ \[ \mathcal{L} \{ e^{at} f(t) \} \] }
    \begin{align*}
      \mathcal{L}\{e^{at}f(t)\} & = \mathcal{L}\{f(t)\} \biggr |_{s\to s-a} \\
                                & = F(s-a)
    \end{align*}
  \end{flashcard}

  %21
  \begin{flashcard}{Piecewise definition of $ \mathcal{U}(t - a) $}
    \[
      \mathcal{U}(t - a) = 
      \begin{cases}
        0, & 0 \leq t < a \\
        1, & t \geq a.
      \end{cases}
    \]
  \end{flashcard}

  %22
  \begin{flashcard}{Graph of $ \mathcal{U}(t - a) $}
    \begin{tikzpicture}
      \draw (-1, 0) -- (2, 0) node[right] {$t$};
      \draw ( 0, -1) -- (0, 2) node[left] {$y(t)$};
      \draw (1.0, 0.1) -- ( 1.0, -0.1) node[below] {$a$};
      \draw (0.1, 1.0) -- (-0.1, 1.0) node[left] {$1$};
      \draw[very thick] (0, 0) -- (1, 0);
      \draw[very thick] (1, 1) -- (2, 1);
      \fill (1,1) circle [radius=0.05];
    \end{tikzpicture}
  \end{flashcard}

  %23
  \begin{flashcard}{Write $f(t)$ using the unit step function.
    \[ f(t) =
       \begin{cases}
          0,    & 0 \leq t < a, \\
          g(t), & a \leq t < b, \\
          0,    & b \leq t.
        \end{cases}
    \]
    }
    \begin{align*}
      f(t) & = g(t) \, \mathcal{U}(t-a) - g(t) \, \mathcal{U}(t-b)\\
           & = g(t)\left(\mathcal{U}(t-a) - \mathcal{U}(t-b)\right)
    \end{align*}
  \end{flashcard}

  %24
  \begin{flashcard}{ \[ \mathcal{L} 
    \left \{ \mathcal{U}(t - a) \right \}  \] }
    \begin{align*}
      \mathcal{L} \left\{ \mathcal{U}(t-a)\right\} 
        & = \mathcal{L}\left\{ \left(1\right)\,\mathcal{U}(t-a)\right\} \\
        & = e^{-as}  \, \mathcal{L}\left\{ 1\right\} \\
        & =\frac{e^{-as}}{s}
    \end{align*}
  \end{flashcard}

  %25
  \begin{flashcard}{ \[ \mathcal{L} 
    \left \{ f(t - a) \, \mathcal{U}(t - a) \right \}  \] }
    \begin{align*}
      \mathcal{L} \left \{ f(t - a) \, \mathcal{U}(t - a) \right \}
        & = e^{-as}  \, \mathcal{L}\left\{ f(t) \right\} \\
        & = e^{-as} F(s)
    \end{align*}
  \end{flashcard}

  %26
  \begin{flashcard}{$ \mathcal{L} 
    \left \{ g(t) \, \mathcal{U}(t - a) \right \} $}
    \[
      \mathcal{L} \left \{ g(t) \, \mathcal{U}(t - a) \right \} = 
        e^{-as} \mathcal{L} \left \{ g(t + a) \right \}
    \]
  \end{flashcard}

  
  % Convolution

  %27
  \begin{flashcard}{ \[ f * g \] }
    \[
      f * g = \int_0^t f(\tau) \, g(t - \tau) \, d\tau
    \]
  \end{flashcard}

  %28 
  \begin{flashcard}{ \[ \mathcal{L} \{ f * g \} \] }
    \[
      \mathcal{L} \{ f * g \} = 
      \mathcal{L} \{ f(t) \} \, \mathcal{L} \{ g(t) \} = F(s) \, G(s)
    \]
  \end{flashcard}


  % Transform of the Dirac Delta Function

  %29
  \begin{flashcard}{ \[ \mathcal{L} \{ \delta( t - a ) \}  \] }
    \[
      \mathcal{L} \{ \delta( t - a ) \} = e^{-as}, \quad a>0.
    \]
  \end{flashcard}



\end{document}
