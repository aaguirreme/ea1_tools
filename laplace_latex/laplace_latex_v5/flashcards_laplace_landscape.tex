%%%%%%%%%%%%%%%%%%%%%%%%%%%%%%%%%%%%%%%%%%%%%%%%%%%%%%%%%%%%%%%%%%%%%%%%%%%%%%%%
%
% Name:   flashcards_laplace
%
% Author: Andres M. Aguirre-Mesa
%         Ph.D student, Mechanical Engineering
%         University of Texas at San Antonio
%
% Date:   03/21/2019
% Update: 02/20/2020
%
% Description:  this is a set of cards on the topic of Laplace Transforms,
%               to be applied for solving ordinary differential equations.
%               The size is of Poker playing cards: 2.5" x 3.5"
%
%%%%%%%%%%%%%%%%%%%%%%%%%%%%%%%%%%%%%%%%%%%%%%%%%%%%%%%%%%%%%%%%%%%%%%%%%%%%%%%%

\documentclass[grid, poker_landscape]{flashcards}

\usepackage{mathtools}
\usepackage{tikz}
%\usepackage{bm}                              % Uncomment for bold math fonts.

\begin{document}

  \cardfrontstyle[\Huge]{plain}
  \cardbackstyle[\Huge]{plain}

  % Definition of the Laplace Transform

  %1
  \begin{flashcard}{ Integral definition of $ \mathcal{L} \{ f(t) \} $ }
    \[
      \int_0^\infty e^{-st} f(t) \, dt
    \]
  \end{flashcard}

  % % This is a version of card #1 with all text in bold fonts.
  % %1
  % \begin{flashcard}{ \textbf{Integral definition of} 
  %   $ \bm{ \mathcal{L} \{ f(t) \} } $ }
  %   \[
  %     \bm{\mathcal{L} \{ f(t) \}  = \int_0^\infty e^{-st} f(t) \, dt}
  %   \]
  % \end{flashcard}

  % Transforms of basic functions

  %2
  \begin{flashcard}{ \[ \mathcal{L} \{ 1 \} \] }
    \[
      \frac{1}{s}
    \]
  \end{flashcard}

  %3
  \begin{flashcard}{ \[ \mathcal{L} \{ t \} \] }
    \[
      \frac{ 1 }{ s^2 }
    \]
  \end{flashcard}

  %4
  \begin{flashcard}{ \[ \mathcal{L} \{ t^n \}, \quad n=1,2,3,\ldots \] }
    \[
      \frac{ n! }{ s^{n + 1} }, \quad n=1,2,3,\ldots
    \]
  \end{flashcard}

  %5 
  \begin{flashcard}{ \[ \mathcal{L} \{ e^{at} \} \] }
    \[
      \frac{ 1 }{ s - a }
    \]
  \end{flashcard}

  %6
  \begin{flashcard}{\[ \mathcal{L} \{ \sin kt \} \] }
    \[
      \frac{ k }{ s^2 + k^2 }
    \]
  \end{flashcard}

  %7
  \begin{flashcard}{ \[ \mathcal{L} \{ \cos kt \} \] }
    \[
      \frac{ s }{ s^2 + k^2 }
    \]
  \end{flashcard}

  %8
  \begin{flashcard}{ \[ \mathcal{L} \{ \sinh kt \} \] }
    \[
      \frac{ k }{ s^2 - k^2 }
    \]
  \end{flashcard}

  %9
  \begin{flashcard}{ \[ \mathcal{L} \{ \cosh kt \} \] }
    \[
      \frac{ s }{ s^2 - k^2 }
    \]
  \end{flashcard}


  % Inverse transforms

  %10
  \begin{flashcard}{ \[ \mathcal{L}^{-1} \left \{ \frac{1}{s} \right \} \] }
    \[
      1
    \]
  \end{flashcard}

  %11
  \begin{flashcard}{ \[ \mathcal{L}^{-1} \left \{ \frac{1}{s^2} \right \} \] }
    \[
      t
    \]
  \end{flashcard}

  %12
  \begin{flashcard}{ \[ \begin{gathered} \mathcal{L}^{-1} 
    \left \{ \frac{ n! }{ s^{n + 1} } \right \}, \\ n=1,2,3,\ldots 
    \end{gathered} \] }
    \[
      t^n, n=1,2,3,\ldots
    \]
  \end{flashcard}

  %13
  \begin{flashcard}{ \[ \mathcal{L}^{-1} 
    \left \{ \frac{ 1 }{ s - a } \right \} \] }
    \[
      e^{at}
    \]
  \end{flashcard}

  %14
  \begin{flashcard}{ \[ \mathcal{L}^{-1} 
    \left \{ \frac{ k }{ s^2 + k^2 } \right \}  \] }
    \[
      \sin kt
    \]
  \end{flashcard}

  %15
  \begin{flashcard}{\[ \mathcal{L}^{-1} 
    \left \{ \frac{ s }{ s^2 + k^2 } \right \} \] }
    \[
      \cos kt
    \]
  \end{flashcard}

  %16
  \begin{flashcard}{ \[ \mathcal{L}^{-1} 
    \left \{ \frac{ k }{ s^2 - k^2 } \right \}  \] }
    \[
      \sinh kt
    \]
  \end{flashcard}

  %17
  \begin{flashcard}{ \[ \mathcal{L}^{-1} 
    \left \{ \frac{ s }{ s^2 - k^2 } \right \} \] }
    \[
      \cosh kt
    \]
  \end{flashcard}


  % Transforms of derivatives

  %18
  \begin{flashcard}{ \[ \mathcal{L} \{ f'(t) \} \] }
    \[
      s \, F(s)-f(0)
    \]
  \end{flashcard}

  %19
  \begin{flashcard}{ \[ \mathcal{L} \{ f''(t) \} \] }
    \[
      s^{2} F(s) -s\,f(0) - f'(0)
    \]
  \end{flashcard}


  % Translations

  %20
  \begin{flashcard}{ \[ \mathcal{L} \{ e^{at} f(t) \} \] }
    \[
      F(s-a)
    \]
  \end{flashcard}

  %21
  \begin{flashcard}{Piecewise definition of $ u(t - a) $}
    \[
        \begin{multlined}
            u(t - a) = \\
            \begin{cases}
                0, & 0 \leq t < a \\
                1, & t \geq a.
            \end{cases}
        \end{multlined}
    \]
  \end{flashcard}

  %22
  \begin{flashcard}{Graph of $ u(t - a) $}
    \begin{tikzpicture}
      \draw (-1,  0) -- (2, 0) node[right] {$t$};
      \draw ( 0, -1) -- (0, 2) node[left] {$y(t)$};
      \draw (1.0, 0.1) -- ( 1.0, -0.1) node[below] {$a$};
      \draw (0.1, 1.0) -- (-0.1, 1.0) node[left] {$1$};
      \draw[very thick] (0, 0) -- (1, 0);
      \draw[very thick] (1, 1) -- (2, 1);
      \fill (1,1) circle [radius=0.05];
    \end{tikzpicture}
  \end{flashcard}

  %23
  \begin{flashcard}{
    \[
        \begin{multlined} 
            f(t) = \\
            \begin{cases}
                  0,    & 0 \leq t < a, \\
                  g(t), & a \leq t < b, \\
                  0,    & b \leq t.
            \end{cases}
        \end{multlined}
    \]
    }
    \[
        \begin{multlined}
            f(t) = g(t) \Big [ u(t-a) \\
              - u(t-b) \Big ]
        \end{multlined}
    \]
  \end{flashcard}

  %24
  \begin{flashcard}{ \[ \mathcal{L} \{ u(t - a) \} \] }
    \[
      \frac{e^{-as}}{s}
    \]
  \end{flashcard}

  %25
  \begin{flashcard}{ \[ \mathcal{L} 
    \left \{ f(t - a) \, u(t - a) \right \}  \] }
    \[
      e^{-as} F(s)
    \]
  \end{flashcard}

  %26
  \begin{flashcard}{ \[ \mathcal{L} 
    \left \{ g(t) \, u(t - a) \right \} \] }
    \[
      e^{-as} \mathcal{L} \left \{ g(t + a) \right \}
    \]
  \end{flashcard}

  
  % Convolution

  %27
  \begin{flashcard}{ Integral definition of $ f * g $ }
    \[
      \int_0^t f(\tau) \, g(t - \tau) \, d\tau
    \]
  \end{flashcard}

  %28 
  \begin{flashcard}{ \[ \mathcal{L} \{ f * g \} \] }
    \[
      F(s) \, G(s)
    \]
  \end{flashcard}


  % Transform of the Dirac Delta Function

  % %29
  % \begin{flashcard}{ \[ \mathcal{L} \{ \delta( t - a ) \}  \] }
  %   \[
  %     \mathcal{L} \{ \delta( t - a ) \} = e^{-as}, \quad a>0.
  %   \]
  % \end{flashcard}

  % Other

  % \begin{flashcard}{ Complete the square \[ a\,s^2 + b\,s + c \] }
  %   \begin{align*}
  %     & a\,s^2 + b\,s + c \\
  %     & = a \left( s^{2} + \tfrac{b}{a}\,s + \tfrac{c}{a} \right) \\
  %     & = a \left[ \left( s + \tfrac{b}{2a}\right)^2 + \tfrac{c}{a} 
  %       - \left( \tfrac{b}{2a} \right)^2 \right]
  %   \end{align*}
  % \end{flashcard}

  % \begin{flashcard}{ \[ \frac{ k\,e^{-a\,s} }{ s^2 + k^2} \] }
  %   \[
  %     u(t - a) \sin(k\,t - k\,a)
  %   \]
  % \end{flashcard}

\end{document}
