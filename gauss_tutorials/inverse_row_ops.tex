%% LyX 2.3.4.2 created this file.  For more info, see http://www.lyx.org/.
%% Do not edit unless you really know what you are doing.
\documentclass[12pt,english]{beamer}
\usepackage[T1]{fontenc}
\usepackage[latin9]{inputenc}
\setcounter{secnumdepth}{3}
\setcounter{tocdepth}{3}
\setlength{\parskip}{\smallskipamount}
\setlength{\parindent}{0pt}
\usepackage{babel}
\usepackage{bm}
\usepackage{amstext}
\ifx\hypersetup\undefined
  \AtBeginDocument{%
    \hypersetup{unicode=true,pdfusetitle,
 bookmarks=true,bookmarksnumbered=false,bookmarksopen=false,
 breaklinks=false,pdfborder={0 0 0},pdfborderstyle={},backref=false,colorlinks=false}
  }
\else
  \hypersetup{unicode=true,pdfusetitle,
 bookmarks=true,bookmarksnumbered=false,bookmarksopen=false,
 breaklinks=false,pdfborder={0 0 0},pdfborderstyle={},backref=false,colorlinks=false}
\fi

\makeatletter
%%%%%%%%%%%%%%%%%%%%%%%%%%%%%% Textclass specific LaTeX commands.
% this default might be overridden by plain title style
\newcommand\makebeamertitle{\frame{\maketitle}}%
% (ERT) argument for the TOC
\AtBeginDocument{%
  \let\origtableofcontents=\tableofcontents
  \def\tableofcontents{\@ifnextchar[{\origtableofcontents}{\gobbletableofcontents}}
  \def\gobbletableofcontents#1{\origtableofcontents}
}

%%%%%%%%%%%%%%%%%%%%%%%%%%%%%% User specified LaTeX commands.
\usepackage{xcolor}

\beamertemplatenavigationsymbolsempty
\setbeamercovered{transparent}

\definecolor{green}{RGB}{0,164,0}

\makeatother

\begin{document}
\begin{frame}{Example problem - inverse by row operations}

Find the inverse of $\bm{A}$ using row operations.

{\scriptsize{}
\[
\bm{A}=\left[\begin{array}{rrr}
0 & 1 & -1\\
1 & 4 & 5\\
1 & 5 & 3
\end{array}\right]
\]
}\vspace{-1em}

Create an augmented matrix with $\bm{A}$ and the identity matrix.
Then use Gauss-Jordan elimination to convert the matrix $\bm{A}$
to the identity matrix. The process will convert the right-hand identitiy
matrix to $\bm{A}^{-1}$.

\vspace{-1em}

{\scriptsize{}
\[
\begin{bmatrix}\begin{array}{rrr|rrr}
0 & 1 & -1 & 1 & 0 & 0\\
1 & 4 & 5 & 0 & 1 & 0\\
1 & 5 & 3 & 0 & 0 & 1
\end{array}\end{bmatrix}\stackrel{\text{Row operations}}{\Longrightarrow}\begin{bmatrix}\begin{array}{rrr|rrr}
1 & 0 & 0 & -13 & -8 & 9\\
0 & 1 & 0 & 2 & 1 & -1\\
0 & 0 & 1 & 1 & 1 & -1
\end{array}\end{bmatrix}
\]
}{\scriptsize\par}

As a check, one can multiply $\bm{A}^{-1}\bm{A}$ or $\bm{A}\bm{A}^{-1}$
to verify that the product is the identity matrix.

{\scriptsize{}
\[
\bm{A}\bm{A}^{-1}=\left[\begin{array}{rrr}
0 & 1 & -1\\
1 & 4 & 5\\
1 & 5 & 3
\end{array}\right]\left[\begin{array}{rrr}
-13 & -8 & 9\\
2 & 1 & -1\\
1 & 1 & -1
\end{array}\right]=\left[\begin{array}{rrr}
1 & 0 & 0\\
0 & 1 & 0\\
0 & 0 & 1
\end{array}\right]=\bm{I}
\]
}{\scriptsize\par}
\end{frame}
%
\begin{frame}{{\small{}Perform Gauss-Jordan elimination}}

For convenience, swap rows 1 and 3 to perform the row elimination
using $a_{11}=1$, and to make $a_{31}=0$.

{\scriptsize{}
\begin{equation}
\begin{bmatrix}\begin{array}{rrr|rcr}
{\color{blue}0} & {\color{blue}1} & {\color{blue}-1} & {\color{blue}1} & {\color{blue}0} & {\color{blue}0}\\
1 & 4 & 5 & 0 & 1 & 0\\
{\color{red}1} & {\color{red}5} & {\color{red}3} & {\color{red}0} & {\color{red}0} & {\color{red}1}
\end{array}\end{bmatrix}\stackrel{{\color{blue}R_{1}}\leftrightarrow{\color{red}R_{3}}}{\Longrightarrow}\begin{bmatrix}\begin{array}{rrr|rcr}
{\color{red}1} & {\color{red}5} & {\color{red}3} & {\color{red}0} & {\color{red}0} & {\color{red}1}\\
1 & 4 & 5 & 0 & 1 & 0\\
{\color{blue}0} & {\color{blue}1} & {\color{blue}-1} & {\color{blue}1} & {\color{blue}0} & {\color{blue}0}
\end{array}\end{bmatrix}
\end{equation}
}{\scriptsize\par}

Make $a_{21}=0$ using information from the first row.

{\scriptsize{}
\begin{equation}
\begin{bmatrix}\begin{array}{rrr|rcc}
{\color{blue}1} & {\color{blue}5} & {\color{blue}3} & {\color{blue}0} & {\color{blue}0} & {\color{blue}1}\\
{\color{red}1} & {\color{red}4} & {\color{red}5} & {\color{red}0} & {\color{red}1} & {\color{red}0}\\
0 & 1 & -1 & 1 & 0 & 0
\end{array}\end{bmatrix}\stackrel{-{\color{blue}R_{1}}+{\color{red}R_{2}}\to{\color{green}R_{2}}}{\Longrightarrow}\begin{bmatrix}\begin{array}{rrr|rrr}
{\color{blue}1} & {\color{blue}5} & {\color{blue}3} & {\color{blue}0} & {\color{blue}0} & {\color{blue}1}\\
{\color{green}0} & {\color{green}-1} & {\color{green}2} & {\color{green}0} & {\color{green}1} & {\color{green}-1}\\
0 & 1 & -1 & 1 & 0 & 0
\end{array}\end{bmatrix}
\end{equation}
}{\scriptsize\par}

Continue the elimination process by making $a_{32}=0$ using information
from the second row. 

{\scriptsize{}
\begin{equation}
\begin{bmatrix}\begin{array}{rrr|rrr}
1 & 5 & 3 & 0 & 0 & 1\\
{\color{blue}0} & {\color{blue}-1} & {\color{blue}2} & {\color{blue}0} & {\color{blue}1} & {\color{blue}-1}\\
{\color{red}0} & {\color{red}1} & {\color{red}-1} & {\color{red}1} & {\color{red}0} & {\color{red}0}
\end{array}\end{bmatrix}\stackrel{{\color{blue}R_{2}}+{\color{red}R_{3}}\to{\color{green}R_{3}}}{\Longrightarrow}\begin{bmatrix}\begin{array}{rrr|rrr}
1 & 5 & 3 & 0 & 0 & 1\\
{\color{blue}0} & {\color{blue}-1} & {\color{blue}2} & {\color{blue}0} & {\color{blue}1} & {\color{blue}-1}\\
{\color{green}0} & {\color{green}0} & {\color{green}1} & {\color{green}1} & {\color{green}1} & {\color{green}-1}
\end{array}\end{bmatrix}
\end{equation}
}{\scriptsize\par}
\end{frame}
%
\begin{frame}{}

To put the augmented matrix in row-echelon form, multiply the second
row by $-1$.

{\scriptsize{}
\begin{equation}
\begin{bmatrix}\begin{array}{rrr|rrr}
1 & 5 & 3 & 0 & 0 & 1\\
{\color{red}0} & {\color{red}-1} & {\color{red}2} & {\color{red}0} & {\color{red}1} & {\color{red}-1}\\
0 & 0 & 1 & 1 & 1 & -1
\end{array}\end{bmatrix}\stackrel{-{\color{red}R_{2}}\to{\color{green}R_{2}}}{\Longrightarrow}\begin{bmatrix}\begin{array}{rrr|rrr}
1 & 5 & 3 & 0 & 0 & 1\\
{\color{green}0} & {\color{green}1} & {\color{green}-2} & {\color{green}0} & {\color{green}-1} & {\color{green}1}\\
0 & 0 & 1 & 1 & 1 & -1
\end{array}\end{bmatrix}
\end{equation}
}{\scriptsize\par}

Make $a_{23}=0$ using information from the third row.

{\scriptsize{}
\begin{equation}
\begin{bmatrix}\begin{array}{rrr|rrr}
1 & 5 & 3 & 0 & 0 & 1\\
{\color{red}0} & {\color{red}1} & {\color{red}-2} & {\color{red}0} & {\color{red}-1} & {\color{red}1}\\
{\color{blue}0} & {\color{blue}0} & {\color{blue}1} & {\color{blue}1} & {\color{blue}1} & {\color{blue}-1}
\end{array}\end{bmatrix}\stackrel{2{\color{blue}R_{3}}+{\color{red}R_{2}}\to{\color{green}R_{2}}}{\Longrightarrow}\begin{bmatrix}\begin{array}{rrr|rrr}
1 & 5 & 3 & 0 & 0 & 1\\
{\color{green}0} & {\color{green}1} & {\color{green}0} & {\color{green}2} & {\color{green}1} & {\color{green}-1}\\
{\color{blue}0} & {\color{blue}0} & {\color{blue}1} & {\color{blue}1} & {\color{blue}1} & {\color{blue}-1}
\end{array}\end{bmatrix}
\end{equation}
}{\scriptsize\par}

Make $a_{13}=0$ using information from the third row.

{\scriptsize{}
\begin{equation}
\begin{bmatrix}\begin{array}{rrr|rrr}
{\color{red}1} & {\color{red}5} & {\color{red}3} & {\color{red}0} & {\color{red}0} & {\color{red}1}\\
0 & 1 & 0 & 2 & 1 & -1\\
{\color{blue}0} & {\color{blue}0} & {\color{blue}1} & {\color{blue}1} & {\color{blue}1} & {\color{blue}-1}
\end{array}\end{bmatrix}\stackrel{-3{\color{blue}R_{3}}+{\color{red}R_{1}}\to{\color{green}R_{1}}}{\Longrightarrow}\begin{bmatrix}\begin{array}{rrr|rrr}
{\color{green}1} & {\color{green}5} & {\color{green}0} & {\color{green}-3} & {\color{green}-3} & {\color{green}4}\\
0 & 1 & 0 & 2 & 1 & -1\\
{\color{blue}0} & {\color{blue}0} & {\color{blue}1} & {\color{blue}1} & {\color{blue}1} & {\color{blue}-1}
\end{array}\end{bmatrix}
\end{equation}
}{\scriptsize\par}
\end{frame}
%
\begin{frame}{}

Make $a_{12}=0$ using information from the second row.

{\scriptsize{}
\begin{equation}
\begin{bmatrix}\begin{array}{rrr|rrr}
{\color{red}1} & {\color{red}5} & {\color{red}0} & {\color{red}-3} & {\color{red}-3} & {\color{red}4}\\
{\color{blue}0} & {\color{blue}1} & {\color{blue}0} & {\color{blue}2} & {\color{blue}1} & {\color{blue}-1}\\
0 & 0 & 1 & 1 & 1 & -1
\end{array}\end{bmatrix}\stackrel{-5{\color{blue}R_{2}}+{\color{red}R_{1}}\to{\color{green}R_{1}}}{\Longrightarrow}\begin{bmatrix}\begin{array}{rrr|rrr}
{\color{green}1} & {\color{green}0} & {\color{green}0} & {\color{green}-13} & {\color{green}-8} & {\color{green}9}\\
{\color{blue}0} & {\color{blue}1} & {\color{blue}0} & {\color{blue}2} & {\color{blue}1} & {\color{blue}-1}\\
0 & 0 & 1 & 1 & 1 & -1
\end{array}\end{bmatrix}
\end{equation}
}{\scriptsize\par}

The augmented matrix is now in reduced row-echelon form, and the inverse
of $\bm{A}$ is now the matrix on the right hand side.

\[
\bm{A}^{-1}=\left[\begin{array}{rrr}
-13 & -8 & 9\\
2 & 1 & -1\\
1 & 1 & -1
\end{array}\right]
\]
\end{frame}

\end{document}
