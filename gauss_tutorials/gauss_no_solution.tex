%% LyX 2.3.5.2 created this file.  For more info, see http://www.lyx.org/.
%% Do not edit unless you really know what you are doing.
\documentclass[12pt,english]{beamer}
\usepackage[T1]{fontenc}
\usepackage[latin9]{inputenc}
\setcounter{secnumdepth}{3}
\setcounter{tocdepth}{3}
\setlength{\parskip}{\smallskipamount}
\setlength{\parindent}{0pt}
\usepackage{babel}
\usepackage{units}
\usepackage{amstext}
\ifx\hypersetup\undefined
  \AtBeginDocument{%
    \hypersetup{unicode=true,pdfusetitle,
 bookmarks=true,bookmarksnumbered=false,bookmarksopen=false,
 breaklinks=false,pdfborder={0 0 0},pdfborderstyle={},backref=false,colorlinks=false}
  }
\else
  \hypersetup{unicode=true,pdfusetitle,
 bookmarks=true,bookmarksnumbered=false,bookmarksopen=false,
 breaklinks=false,pdfborder={0 0 0},pdfborderstyle={},backref=false,colorlinks=false}
\fi

\makeatletter
%%%%%%%%%%%%%%%%%%%%%%%%%%%%%% Textclass specific LaTeX commands.
% this default might be overridden by plain title style
\newcommand\makebeamertitle{\frame{\maketitle}}%
% (ERT) argument for the TOC
\AtBeginDocument{%
  \let\origtableofcontents=\tableofcontents
  \def\tableofcontents{\@ifnextchar[{\origtableofcontents}{\gobbletableofcontents}}
  \def\gobbletableofcontents#1{\origtableofcontents}
}

%%%%%%%%%%%%%%%%%%%%%%%%%%%%%% User specified LaTeX commands.
\usepackage{xcolor}

\beamertemplatenavigationsymbolsempty
\setbeamercovered{transparent}

\definecolor{green}{RGB}{0,164,0}

\makeatother

\begin{document}
\begin{frame}{Example problem - no solution}

Solve the system of equations:

\vspace{-1em}

\begin{alignat*}{4}
4x_{1} & \;-\; & 7x_{2} & \;-\; & 2x_{3} & \;=\; &  & 1\\
3x_{1} & \;-\; & 7x_{2} & \;-\; & 5x_{3} & \;=\; & - & 6\\
x_{1} & \;-\; & x_{2} & \;+\; & x_{3} & \;=\; &  & 2
\end{alignat*}

The process can be summarized as:

\[
\begin{bmatrix}\begin{array}{rrr|r}
4 & -7 & -2 & 1\\
3 & -7 & -5 & -6\\
1 & -1 & 1 & 2
\end{array}\end{bmatrix}\stackrel{\text{Row operations}}{\Longrightarrow}\begin{bmatrix}\begin{array}{rrr|r}
1 & -1 & 1 & 2\\
0 & 1 & 2 & 3\\
0 & 0 & 0 & 2
\end{array}\end{bmatrix}
\]

\end{frame}
%
\begin{frame}{}

For convenience, swap rows 1 and 3 to perform next operations with
integers instead of fractions.

{\small{}
\begin{equation}
\begin{bmatrix}\begin{array}{rrr|r}
{\color{blue}4} & {\color{blue}-7} & {\color{blue}-2} & {\color{blue}1}\\
3 & -7 & -5 & -6\\
{\color{red}1} & {\color{red}-1} & {\color{red}1} & {\color{red}2}
\end{array}\end{bmatrix}\stackrel{{\color{blue}R_{1}}\leftrightarrow{\color{red}R_{3}}}{\Longrightarrow}\begin{bmatrix}\begin{array}{rrr|r}
{\color{red}1} & {\color{red}-1} & {\color{red}1} & {\color{red}2}\\
3 & -7 & -5 & -6\\
{\color{blue}4} & {\color{blue}-7} & {\color{blue}-2} & {\color{blue}1}
\end{array}\end{bmatrix}
\end{equation}
}{\small\par}

Make $a_{21}=0$ using information from the first row.

{\small{}
\begin{equation}
\begin{bmatrix}\begin{array}{rrr|r}
{\color{blue}1} & {\color{blue}-1} & {\color{blue}1} & {\color{blue}2}\\
{\color{red}3} & {\color{red}-7} & {\color{red}-5} & {\color{red}-6}\\
4 & -7 & -2 & 1
\end{array}\end{bmatrix}\stackrel{-3{\color{blue}R_{1}}+{\color{red}R_{2}}\to{\color{green}R_{2}}}{\Longrightarrow}\begin{bmatrix}\begin{array}{rrr|r}
{\color{blue}1} & {\color{blue}-1} & {\color{blue}1} & {\color{blue}2}\\
{\color{green}0} & {\color{green}-4} & {\color{green}-8} & {\color{green}-12}\\
4 & -7 & -2 & 1
\end{array}\end{bmatrix}
\end{equation}
}{\small\par}

Make $a_{31}=0$ using information from the first row.

{\small{}
\begin{equation}
\begin{bmatrix}\begin{array}{rrr|r}
{\color{blue}1} & {\color{blue}-1} & {\color{blue}1} & {\color{blue}2}\\
0 & -4 & -8 & -12\\
{\color{red}4} & {\color{red}-7} & {\color{red}-2} & {\color{red}1}
\end{array}\end{bmatrix}\stackrel{-4{\color{blue}R_{1}}+{\color{red}R_{3}}\to{\color{green}R_{3}}}{\Longrightarrow}\begin{bmatrix}\begin{array}{rrr|r}
{\color{blue}1} & {\color{blue}-1} & {\color{blue}1} & {\color{blue}2}\\
0 & -4 & -8 & -12\\
{\color{green}0} & {\color{green}-3} & {\color{green}-6} & {\color{green}-7}
\end{array}\end{bmatrix}
\end{equation}
}{\small\par}
\end{frame}
%
\begin{frame}{}

For convenience, divide the second row by $-4$ to perform next operations
with integers instead of fractions.

\begin{equation}
\begin{bmatrix}\begin{array}{rrr|r}
1 & -1 & 1 & 2\\
{\color{red}0} & {\color{red}-4} & {\color{red}-8} & {\color{red}-12}\\
0 & -3 & -6 & -7
\end{array}\end{bmatrix}\stackrel{-\nicefrac{1}{4}{\color{red}R_{2}}\to{\color{green}R_{2}}}{\Longrightarrow}\begin{bmatrix}\begin{array}{rrr|r}
1 & -1 & 1 & 2\\
{\color{green}0} & {\color{green}1} & {\color{green}2} & {\color{green}3}\\
0 & -3 & -6 & -7
\end{array}\end{bmatrix}
\end{equation}

Make $a_{32}=0$ using information from the second row.

\begin{equation}
\begin{bmatrix}\begin{array}{rrr|r}
1 & -1 & 1 & 2\\
{\color{blue}0} & {\color{blue}1} & {\color{blue}2} & {\color{blue}3}\\
{\color{red}0} & {\color{red}-3} & {\color{red}-6} & {\color{red}-7}
\end{array}\end{bmatrix}\stackrel{3{\color{blue}R_{2}}+{\color{red}R_{3}}\to{\color{green}R_{3}}}{\Longrightarrow}\begin{bmatrix}\begin{array}{rrr|r}
1 & -1 & 1 & 2\\
{\color{blue}0} & {\color{blue}1} & {\color{blue}2} & {\color{blue}3}\\
{\color{green}0} & {\color{green}0} & {\color{green}0} & {\color{green}2}
\end{array}\end{bmatrix}
\end{equation}

\end{frame}
%
\begin{frame}{}

\[
\begin{bmatrix}\begin{array}{rrr|r}
1 & -1 & 1 & 2\\
0 & 1 & 2 & 3\\
0 & 0 & 0 & 2
\end{array}\end{bmatrix}
\]

\medskip{}

The matrix is now in upper triangular form. However, notice that the
last row contains a contradiction. It reads $0x_{1}+0x_{2}+0x_{3}=2$,
or simply $0=2$.

This does not make mathematical sense, so the system is called inconsistent,
and has no solution. 
\end{frame}

\end{document}
