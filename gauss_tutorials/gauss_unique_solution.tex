%% LyX 2.3.4.2 created this file.  For more info, see http://www.lyx.org/.
%% Do not edit unless you really know what you are doing.
\documentclass[12pt,english]{beamer}
\usepackage[T1]{fontenc}
\usepackage[latin9]{inputenc}
\setcounter{secnumdepth}{3}
\setcounter{tocdepth}{3}
\setlength{\parskip}{\smallskipamount}
\setlength{\parindent}{0pt}
\usepackage{babel}
\usepackage{units}
\usepackage{amstext}
\ifx\hypersetup\undefined
  \AtBeginDocument{%
    \hypersetup{unicode=true,pdfusetitle,
 bookmarks=true,bookmarksnumbered=false,bookmarksopen=false,
 breaklinks=false,pdfborder={0 0 0},pdfborderstyle={},backref=false,colorlinks=false}
  }
\else
  \hypersetup{unicode=true,pdfusetitle,
 bookmarks=true,bookmarksnumbered=false,bookmarksopen=false,
 breaklinks=false,pdfborder={0 0 0},pdfborderstyle={},backref=false,colorlinks=false}
\fi

\makeatletter
%%%%%%%%%%%%%%%%%%%%%%%%%%%%%% Textclass specific LaTeX commands.
% this default might be overridden by plain title style
\newcommand\makebeamertitle{\frame{\maketitle}}%
% (ERT) argument for the TOC
\AtBeginDocument{%
  \let\origtableofcontents=\tableofcontents
  \def\tableofcontents{\@ifnextchar[{\origtableofcontents}{\gobbletableofcontents}}
  \def\gobbletableofcontents#1{\origtableofcontents}
}

%%%%%%%%%%%%%%%%%%%%%%%%%%%%%% User specified LaTeX commands.
\usepackage{xcolor}

\beamertemplatenavigationsymbolsempty
\setbeamercovered{transparent}

\definecolor{green}{RGB}{0,164,0}

\makeatother

\begin{document}
\begin{frame}{Example problem - unique solution}

Solve the system of equations:

\vspace{-1em}

\begin{alignat*}{4}
4x_{1} & \;+\; & x_{2} & \;-\; & 2x_{3} & \;=\; & 3\\
3x_{1} & \;-\; & x_{2} & \;+\; & x_{3} & \;=\; & 2\\
x_{1} & \;-\; & x_{2} & \;+\; & x_{3} & \;=\; & 0
\end{alignat*}

The process can be summarized as:

\[
\begin{bmatrix}\begin{array}{rrr|r}
4 & 1 & -2 & 3\\
3 & -1 & 1 & 2\\
1 & -1 & 1 & 0
\end{array}\end{bmatrix}\stackrel{\text{Row operations}}{\Longrightarrow}\begin{bmatrix}\begin{array}{rrr|r}
1 & 0 & 0 & 1\\
0 & 1 & 0 & 3\\
0 & 0 & 1 & 2
\end{array}\end{bmatrix}
\]

\end{frame}
%
\begin{frame}{}

For convenience, swap rows 1 and 3 to perform next operations with
integers instead of fractions.

\begin{equation}
\begin{bmatrix}\begin{array}{rrr|r}
{\color{blue}4} & {\color{blue}1} & {\color{blue}-2} & {\color{blue}3}\\
3 & -1 & 1 & 2\\
{\color{red}1} & {\color{red}-1} & {\color{red}1} & {\color{red}0}
\end{array}\end{bmatrix}\stackrel{{\color{blue}R_{1}}\leftrightarrow{\color{red}R_{3}}}{\Longrightarrow}\begin{bmatrix}\begin{array}{rrr|r}
{\color{red}1} & {\color{red}-1} & {\color{red}1} & {\color{red}0}\\
3 & -1 & 1 & 2\\
{\color{blue}4} & {\color{blue}1} & {\color{blue}-2} & {\color{blue}3}
\end{array}\end{bmatrix}
\end{equation}

Make $a_{21}=0$ using information from the first row.

\begin{equation}
\begin{bmatrix}\begin{array}{rrr|r}
{\color{blue}1} & {\color{blue}-1} & {\color{blue}1} & {\color{blue}0}\\
{\color{red}3} & {\color{red}-1} & {\color{red}1} & {\color{red}2}\\
4 & 1 & -2 & 3
\end{array}\end{bmatrix}\stackrel{-3{\color{blue}R_{1}}+{\color{red}R_{2}}\to{\color{green}R_{2}}}{\Longrightarrow}\begin{bmatrix}\begin{array}{rrr|r}
{\color{blue}1} & {\color{blue}-1} & {\color{blue}1} & {\color{blue}0}\\
{\color{green}0} & {\color{green}2} & {\color{green}-2} & {\color{green}2}\\
4 & 1 & -2 & 3
\end{array}\end{bmatrix}
\end{equation}

Make $a_{31}=0$ using information from the first row.

\begin{equation}
\begin{bmatrix}\begin{array}{rrr|r}
{\color{blue}1} & {\color{blue}-1} & {\color{blue}1} & {\color{blue}0}\\
0 & 2 & -2 & 2\\
{\color{red}4} & {\color{red}1} & {\color{red}-2} & {\color{red}3}
\end{array}\end{bmatrix}\stackrel{-4{\color{blue}R_{1}}+{\color{red}R_{3}}\to{\color{green}R_{3}}}{\Longrightarrow}\begin{bmatrix}\begin{array}{rrr|r}
{\color{blue}1} & {\color{blue}-1} & {\color{blue}1} & {\color{blue}0}\\
0 & 2 & -2 & 2\\
{\color{green}0} & {\color{green}5} & {\color{green}-6} & {\color{green}3}
\end{array}\end{bmatrix}
\end{equation}

\end{frame}
%
\begin{frame}{}

For convenience, divide the second row by $2$ to perform next operations
with integers instead of fractions.

\begin{equation}
\begin{bmatrix}\begin{array}{rrr|r}
1 & -1 & 1 & 0\\
{\color{red}0} & {\color{red}2} & {\color{red}-2} & {\color{red}2}\\
0 & 5 & -6 & 3
\end{array}\end{bmatrix}\stackrel{-\nicefrac{1}{2}{\color{red}R}_{{\color{red}2}}\to{\color{green}R_{2}}}{\Longrightarrow}\begin{bmatrix}\begin{array}{rrr|r}
1 & -1 & 1 & 0\\
{\color{green}0} & {\color{green}1} & {\color{green}-1} & {\color{green}1}\\
0 & 5 & -6 & 3
\end{array}\end{bmatrix}
\end{equation}

Make $a_{32}=0$ using information from the second row.

\begin{equation}
\begin{bmatrix}\begin{array}{rrr|r}
1 & -1 & 1 & 0\\
{\color{blue}0} & {\color{blue}1} & {\color{blue}-1} & {\color{blue}1}\\
{\color{red}0} & {\color{red}5} & {\color{red}-6} & {\color{red}3}
\end{array}\end{bmatrix}\stackrel{-5{\color{blue}R_{2}}+{\color{red}R_{3}}\to{\color{green}R_{3}}}{\Longrightarrow}\begin{bmatrix}\begin{array}{rrr|r}
1 & -1 & 1 & 0\\
{\color{blue}0} & {\color{blue}1} & {\color{blue}-1} & {\color{blue}1}\\
{\color{green}0} & {\color{green}0} & {\color{green}-1} & {\color{green}-2}
\end{array}\end{bmatrix}
\end{equation}

Operate the last row so that it translates to $z=2$.

\begin{equation}
\begin{bmatrix}\begin{array}{rrr|r}
1 & -1 & 1 & 0\\
0 & 1 & -1 & 1\\
{\color{red}0} & {\color{red}0} & {\color{red}-1} & {\color{red}-2}
\end{array}\end{bmatrix}\overset{-1{\color{red}R_{3}}\to{\color{green}R_{3}}}{\Longrightarrow}\begin{bmatrix}\begin{array}{rrr|r}
1 & -1 & 1 & 0\\
0 & 1 & -1 & 1\\
{\color{green}0} & {\color{green}0} & {\color{green}1} & {\color{green}2}
\end{array}\end{bmatrix}
\end{equation}

\end{frame}
%
\begin{frame}{}

\[
\begin{bmatrix}\begin{array}{rrr|r}
1 & -1 & 1 & 0\\
0 & 1 & -1 & 1\\
0 & 0 & 1 & 2
\end{array}\end{bmatrix}
\]

\medskip{}

The matrix is now in upper triangular form. Hence the unknown values
can be determined using back substitution.
\begin{enumerate}[a.]
\item Starting from row 3, $x_{3}=2$.
\item Using $x_{3}=2$ in row 2, $x_{2}=1+x_{3}=3$.
\item Using $x_{3}=2$ and $x_{2}=3$ in row 1, $x_{1}=0+x_{2}-x_{3}=1$.
\end{enumerate}
\medskip{}

Alternately, using Gauss-Jordan elimination, the left-hand side matrix
can be reduced to an identity matrix from which the solution can be
determined by inspection.
\end{frame}
%
\begin{frame}{}

Make $a_{23}=0$ using information from the third row.

\begin{equation}
\begin{bmatrix}\begin{array}{rrr|r}
1 & -1 & 1 & 0\\
{\color{red}0} & {\color{red}1} & {\color{red}-1} & {\color{red}1}\\
{\color{blue}0} & {\color{blue}0} & {\color{blue}1} & {\color{blue}2}
\end{array}\end{bmatrix}\stackrel{{\color{blue}R_{3}}+{\color{red}R_{2}}\to{\color{green}R_{2}}}{\Longrightarrow}\begin{bmatrix}\begin{array}{rrr|r}
1 & -1 & 1 & 0\\
{\color{green}0} & {\color{green}1} & {\color{green}0} & {\color{green}3}\\
{\color{blue}0} & {\color{blue}0} & {\color{blue}1} & {\color{blue}2}
\end{array}\end{bmatrix}
\end{equation}

Make $a_{13}=0$ using information from the third row.

\begin{equation}
\begin{bmatrix}\begin{array}{rrr|r}
{\color{red}1} & {\color{red}-1} & {\color{red}1} & {\color{red}0}\\
0 & 1 & 0 & 3\\
{\color{blue}0} & {\color{blue}0} & {\color{blue}1} & {\color{blue}2}
\end{array}\end{bmatrix}\stackrel{-{\color{blue}R_{3}}+{\color{red}R_{1}}\to{\color{green}R_{1}}}{\Longrightarrow}\begin{bmatrix}\begin{array}{rrr|r}
{\color{green}1} & {\color{green}-1} & {\color{green}0} & {\color{green}-2}\\
0 & 1 & 0 & 3\\
{\color{blue}0} & {\color{blue}0} & {\color{blue}1} & {\color{blue}2}
\end{array}\end{bmatrix}
\end{equation}

Make $a_{12}=0$ using information from the second row.

\begin{equation}
\begin{bmatrix}\begin{array}{rrr|r}
{\color{red}1} & {\color{red}-1} & {\color{red}0} & {\color{red}-2}\\
{\color{blue}0} & {\color{blue}1} & {\color{blue}0} & {\color{blue}3}\\
0 & 0 & 1 & 2
\end{array}\end{bmatrix}\stackrel{{\color{blue}R_{2}}+{\color{red}R_{1}}\to{\color{green}R_{1}}}{\Longrightarrow}\begin{bmatrix}\begin{array}{rrr|r}
{\color{green}1} & {\color{green}0} & {\color{green}0} & {\color{green}1}\\
{\color{blue}0} & {\color{blue}1} & {\color{blue}0} & {\color{blue}3}\\
0 & 0 & 1 & 2
\end{array}\end{bmatrix}
\end{equation}
\end{frame}

\end{document}
